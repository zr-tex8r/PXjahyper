% 文字コードは UTF-8
% platex で組版する
\documentclass[a4paper]{jsarticle}
\usepackage{shortvrb}
\MakeShortVerb{\|}
\newcommand{\PkgVersion}{0.4}
\newcommand{\PkgDate}{2019/10/25}
\newcommand{\Pkg}[1]{\textsf{#1}}
\newcommand{\Meta}[1]{$\langle$\mbox{}#1\mbox{}$\rangle$}
\newcommand{\Note}{\par\noindent ※}
\newcommand{\Means}{~:\quad}
\providecommand{\pTeX}{p\TeX}
\providecommand{\upTeX}{u\pTeX}
\providecommand{\pLaTeX}{p\LaTeX}
\providecommand{\upLaTeX}{u\pLaTeX}
%-----------------------------------------------------------
\begin{document}
\title{\Pkg{pxjahyper} パッケージ}
\author{八登崇之\ (Takayuki YATO; aka.~``ZR'')}
\date{v\PkgVersion\quad[\PkgDate]}
\maketitle

%===========================================================
\section{概要}

(u){\pLaTeX} + hyperref + dvipdfmxの組み合わせで
日本語を含む「しおり」をもつPDF文書を作成する
場合に必要となる機能を提供する。
\begin{itemize}
\item dvipdfmx用の「tounicode special」について、
  内部漢字コードに応じて適切なものを出力する。
\item PDF文字列の中でLICR(|\"a| や |\textsection| 等の
  文字出力の命令)が正しく機能するようにする。
  ただし、エンジンが {\pTeX} の場合は、out2uni を利用
  する場合を除き、JIS~X~0208にない文字は出力できない
  (hyperrefの警告が出る)。
\item {\TeX} の版面拡大機能が使われている
  (|\mag| が1000でない;典型的には\Pkg{jsclasses}の
  文書クラスで基底フォントサイズが10\,pt以外である)
  場合に、\Pkg{hyperref}が出力するpapersize specialの
  紙面サイズの値が不正になるが、この場合に |\mag| 値を
  考慮して求めた正しいサイズによるpapersize specialを
  改めて出力する。
\end{itemize}

%===========================================================
\section{パッケージの読込}

|\usepackage| で読み込む。
\begin{quote}\small\begin{verbatim}
\usepackage[オプション,...]{pxjahyper}
\end{verbatim}\end{quote}

使用可能なオプションは以下の通り。
\begin{itemize}
\item |tounicode|(既定)\Means
  dvipdfmx用の「tounicode special」を出力する。
\item |notounicode|\Means
  |tounicode| の否定。
\item |out2uni|\Means
  out2uniフィルタ(角藤氏製作)を使うことを前提にした
  出力を行う。
  (|tounicode| が無効になる。)
  {\upLaTeX} では使用不可。
\item |noout2uni|(既定)\Means
  |out2uni| の否定。
\item |otfcid|(既定)\Means
  \Pkg{japanese-otf}パッケージの |\CID| をPDF文字列中で
  使えるようにする。
  具体的には、当該のAJ1のグリフに“対応”するUnicode文字が
  あればそれを出力し、なければ(警告を出した上で)削除する。
\item |nootfcid| \Means
  |otfcid| の否定。
\item |bigcode|(既定)\Means
  {\upTeX}でのToUnicode CMapとして既定のUTF8-UCSの代わりに\ 
  UTF8-UTF16を用いる。
  (当該のファイルが存在する必要がある。)
\item |nobigcode|\Means
  |bigcode| の否定。
  \Note 0.3a版より既定を |bigcode| に変更した。
\item |dvipdfmx|\Means
  dvipdfmxを前提とした動作を行う。
\item |nodriver|/|none|\Means
  dvipdfmxを前提とした動作を抑止する。
  現状では、この場合には本パッケージは実質的に何の動作も行わない。
\item |auto|(既定)\Means
  \Pkg{hyperref}のドライバがdvipdfmx用ならば |dvipdfmx|、
  それ以外は |none| の動作。
\end{itemize}

%===========================================================
\section{機能}

「概要」で述べた機能は(オプション設定に応じて)
自動的に実施される。

\paragraph{Unicode符号値による入力}
PDF文字列入力中で、|\Ux| が以下の意味になる。
PDF文字列以外では |\Ux| は以前の定義(または未定義)に戻る。

\begin{itemize}
\item |\Ux{|Unicode符号値|}|\Means
  その符号値の文字を出力する。
\end{itemize}

符号値は16進数で指定する。

なお、\Pkg{japanese-otf}パッケージの |\UTF| 命令は、PDF文字列中では
out2uni用の出力を行うように設計されているが、
本パッケージを {\upLaTeX} で用いた場合は、
|\UTF| も(PDF文字列中では)|\Ux| と同じ動作
(つまりtounicode用の出力)になるように変更される。

\paragraph{PDF文字列用の文字命令の定義}
以下の命令が提供される。(プリアンブルでのみ使用可能。)

\begin{itemize}
\item |\pxDeclarePdfTextCommand{\制御綴}{|\Meta{JIS符号値}|}{|\Meta
{Unicode符号値}|}|\Means
  PDF文字列中の |\制御綴| の動作として、
  指定した符号値の文字を出力することを指定する。
\item |\pxDeclarePdfTextComposite{\制御綴}{|\Meta{引数}|}{|\Meta
{JIS符号値}|}{|\Meta{Unicode符号値}|}|\Means
  PDF文字列中の |\制御綴|(アクセント命令)+ \Meta{引数}の
  動作として、指定した符号値の文字を出力することを指定する。
\end{itemize}

これらの命令において、符号値は16進数で指定する。
「JIS符号値」は {\upLaTeX} では使われないので省略して
(空にして)もよい
(或いはそもそも JIS~X~0208 にない文字の場合は省略する)。
逆に「Unicode符号値」は {\pLaTeX} の動作でかつ「JIS符号値」が
指定されている場合は省略してよい。

例えば、以下のように定義しておくと、
PDF文字列中で |\textschwa|(schwa記号)や |\d{t}|(\d{t})が
使えるようになる。
\begin{quote}\small\begin{verbatim}
\pxDeclarePdfTextCommand{\textschwa}{}{0259}
\pxDeclarePdfTextComposite{\d}{t}{}{1E6D}
\end{verbatim}\end{quote}

\end{document}
