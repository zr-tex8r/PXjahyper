% 文字コードは UTF-8
% platex で組版する
\documentclass[a4paper]{jsarticle}
\usepackage{shortvrb}
\MakeShortVerb{\|}
\newcommand{\PkgVersion}{0.2}
\newcommand{\Pkg}[1]{\textsf{#1}}
\newcommand{\Meta}[1]{$\langle$\mbox{}#1\mbox{}$\rangle$}
\newcommand{\Note}{\par\noindent ※}
\newcommand{\Means}{~:\quad}
\providecommand{\pTeX}{p\TeX}
\providecommand{\upTeX}{u\pTeX}
\providecommand{\pLaTeX}{p\LaTeX}
\providecommand{\upLaTeX}{u\pLaTeX}
%-----------------------------------------------------------
\begin{document}
\title{\Pkg{pxjahyper} パッケージ(v\PkgVersion)}
\author{八登崇之\ (Takayuki YATO; aka.~``ZR'')}
\date{2012/05/27}
\maketitle

%===========================================================
\section{概要}

pLaTeX + hyperref + dvipdfmxの組み合わせで
日本語を含む「しおり」をもつPDF文書を作成する
場合に必要となる機能を提供する。
\begin{itemize}
\item dvipdfmx用の「tounicode special」について、
  内部漢字コードに応じて適切なものを出力する。
\item PDF文字列の中でLICR(|\"a| や |\textsection| 等の
  文字出力の命令)が正しく機能するようにする。
  ただし、エンジンが {\pTeX} の場合は、out2uni を利用
  する場合を除き、JIS~X~0208にない文字は出力できない
  (hyperrefの警告が出る)。
\end{itemize}

\paragraph{前提フォーマット}
{\pLaTeX} および {\upLaTeX}。

\paragraph{依存パッケージ}
\begin{itemize}
\item \Pkg{hyperref} パッケージ
\item \Pkg{hyperref} が依存するパッケージ
  (\Pkg{atbegshi} 等)。
\end{itemize}

%===========================================================
\section{パッケージの読込}

|\usepackage| で読み込む。
\begin{verbatim}
\usepackage[オプション,...]{pxjahyper}
\end{verbatim}

使用可能なオプションは以下の通り。
\begin{itemize}
\item |tounicode|(既定)\Means
  dvipdfmx用の「tounicode special」を出力する。
\item |notounicode|\Means
  |tounicode| の否定。
\item |out2uni|\Means
  out2uniフィルタ(角藤氏製作)を使うことを前提にした
  出力を行う。
\item |noout2uni|(既定)\Means
  |out2uni| の否定。
\item |bigcode|\Means
  {\upTeX}でのToUnicode CMapとして既定の UTF8-UCS の代わりに\ 
  UTF8-UTF16 を用いる。
  (当該のファイルが存在する必要がある。)
\item |nobigcode|(既定)\Means
  |bigcode| の否定。
\item |dvipdfmx|\Means
  dvipdfmxを前提とした動作を行う。
\item |none|\Means
  dvipdfmxを前提とした動作を抑止する。
  現状では、この場合には本パッケージは実質的に何の動作も行わない。
\item |auto|(既定)\Means
  \Pkg{hyperref}のドライバがdvipdfmx用ならば |dvipdfmx|、
  それ以外は |none| の動作。
\end{itemize}

%===========================================================
\section{機能}

「概要」で述べた機能は(オプション設定に応じて)
自動的に実施される。

以下の命令が提供される。

\begin{itemize}
\item |\pxDeclarePdfTextCommand{\制御綴}{|\Meta{JIS符号値}|}{|\Meta
{Unicode符号値}|}|\Means
  PDF文字列中の |\制御綴| の動作として、
  指定した符号値の文字を出力することを指定する。
\item |\pxDeclarePdfTextComposite{\制御綴}{|\Meta{引数}|}{|\Meta
{JIS符号値}|}{|\Meta{Unicode符号値}|}|\Means
  PDF文字列中の |\制御綴|(アクセント命令)+ \Meta{引数}の
  動作として、指定した符号値の文字を出力することを指定する。
\end{itemize}

これらの命令において、符号値は16進数で指定する。
「JIS符号値」は {\upLaTeX} では使われないので省略して
(空にして)もよい
(或いはそもそも JIS~X~0208 にない文字の場合は省略する)。
逆に「Unicode符号値」は {\pLaTeX} の動作でかつ「JIS符号値」が
指定されている場合は省略してよい。

例えば、以下のように定義しておくと、
PDF文字列中で |\textschwa|(schwa記号)や |\d{t}|(\d{t})が
使えるようになる。
\begin{quote}\small\begin{verbatim}
\pxDeclarePdfTextCommand{\textschwa}{}{0259}
\pxDeclarePdfTextComposite{\d}{t}{}{1E6D}
\end{verbatim}\end{quote}

\end{document}
